% Created 2025-12-09 Tue 19:10
% Intended LaTeX compiler: pdflatex
\documentclass[a4paper,14pt]{extarticle}
\usepackage{graphicx}
\usepackage{longtable}
\usepackage{wrapfig}
\usepackage{rotating}
\usepackage[normalem]{ulem}
\usepackage{amsmath}
\usepackage{amssymb}
\usepackage{capt-of}
\usepackage[english, main=russian, russian]{babel}
\usepackage{fontspec}
\usepackage{polyglossia}
\setdefaultlanguage{russian}
\setotherlanguage{english}
\setmainfont{CMU Serif}
\setsansfont{CMU Sans Serif}
\setmonofont{CMU Typewriter Text}
\usepackage{svg}
\usepackage{xcolor}
\usepackage{caption}
\newcommand{\incsvg}[2][]{\begin{center}\includesvg[width=\textwidth]{#2}\captionof{figure}{#1}\end{center}}
\usepackage{titlesec}
\titleformat{\section}{\normalfont\large\bfseries}{\thesection}{1em}{}
\usepackage{multicol}
\setlength{\columnsep}{1cm}
\setlength{\parindent}{0pt}
\usepackage[left=1.2cm,right=1.2cm,top=1cm,bottom=2cm]{geometry}
\linespread{1}
\usepackage{hyperref}
\usepackage{amsmath}
\usepackage{mathtools}
\usepackage{array}
\usepackage{float}
\restylefloat{table}
\renewcommand{\arraystretch}{1.7}
\date{\today}
\title{}
\hypersetup{
 pdfauthor={},
 pdftitle={},
 pdfkeywords={},
 pdfsubject={},
 pdfcreator={Emacs 29.3 (Org mode 9.8-pre)}, 
 pdflang={Russian}}
\begin{document}

\begin{center}
МФТИ\\

\vspace{6cm}



\begin{large}
\textbf{Отечет по практическим заданиям} \\
\textbf{1b и 2b}


\vspace{3cm}

\textbf{Филак Александр}

\end{large}
\vspace{9cm}


\vspace{2cm}

Долгопрудный \\
2025
\pagebreak
\end{center}
\section{ОТЧЁТ ПО ПРАКТИЧЕСКИМ ЗАДАНИЯМ 2 И 4}
\label{sec:org9c52ccd}

\subsection{Введение}
\label{sec:orgcd06e18}

В рамках двух практических заданий были изучены и реализованы методы работы с облаками точек, включая визуализацию с использованием цветовой информации и обработку скалярных полей. Задание 2 было посвящено различным способам назначения цветов точкам и их визуализации в 2D и 3D. Задание 4 охватывало работу со скалярными полями: их создание, преобразование, фильтрацию и визуализацию. Оба задания демонстрируют важные аспекты предобработки и анализа трёхмерных данных.
\subsection{Теоретическая часть}
\label{sec:org4a0d4e3}

\subsubsection{Облако точек}
\label{sec:org21cd31c}

Облако точек — это набор точек в трёхмерном пространстве, каждая из которых имеет координаты (X, Y, Z). В дополнение к координатам точки могут содержать дополнительную информацию: цвет (RGB), интенсивность, нормали или произвольные скалярные поля. Облака точек широко используются в компьютерном зрении, робототехнике, картографии и 3D-моделировании.
\subsubsection{Цветовое представление RGB}
\label{sec:orga43d80b}

Цвет в модели RGB представляется тремя компонентами: красным (R), зелёным (G) и синим (B). Каждый компонент может принимать значения от 0 до 1 (нормализованная форма) или от 0 до 255 (целочисленная форма). Комбинация этих компонентов создаёт итоговый цвет точки.
\subsubsection{Скалярные поля}
\label{sec:org28e484f}

Скалярное поле — это функция, сопоставляющая каждой точке пространства некоторое скалярное значение. В контексте облаков точек скалярное поле обычно представлено как дополнительный столбец данных, связанный с каждой точкой. Примеры скалярных полей: высота, температура, интенсивность отражения лидара, расстояние до центра и т.д.
\subsection{Задание 2: Работа с цветами в облаках точек}
\label{sec:orgc0eb7fc}

\subsubsection{Постановка задачи}
\label{sec:org121ce49}

Цель задания — освоить методы визуализации облаков точек с цветовой информацией, научиться назначать цвета точкам на основе их координат и других параметров, сравнить возможности статичной (Matplotlib) и интерактивной (Plotly) визуализации.
\subsubsection{Ход выполнения}
\label{sec:org8d56385}

\begin{enumerate}
\item 1. Генерация облака точек
\label{sec:org8678194}

Было создано равномерно распределённое облако из 1000 точек в кубе [0, 1]\textsuperscript{3}.
\item 2. Назначение цветов по координатам
\label{sec:org5da6f0e}

Цвет каждой точке был назначен на основе её координат:
\begin{itemize}
\item Красный компонент = координата X
\item Зелёный компонент = координата Y
\item Синий компонент = координата Z
\end{itemize}

\begin{figure}[htbp]
\centering
\includegraphics[width=.9\linewidth]{./img/task2_vis_matplotlib.jpg}
\caption{2D-визуализация в Matplotlib (RGB = XYZ)}
\end{figure}
\item 3. 3D-визуализация в Plotly
\label{sec:orga4acaf3}

Создана интерактивная 3D-визуализация, позволяющая вращать сцену, приближать и удалять объект.

\begin{figure}[htbp]
\centering
\includegraphics[width=.9\linewidth]{./img/task2_visualization_plotly.png}
\caption{3D-визуализация в Plotly}
\end{figure}
\item 4. Альтернативная схема раскраски
\label{sec:org8e32436}

Реализована схема раскраски на основе расстояния от точки до центра куба (0.5, 0.5, 0.5). Точки, близкие к центру, окрашены в холодные тона (синий), а удалённые — в тёплые (красный/жёлтый).

\begin{figure}[htbp]
\centering
\includegraphics[width=.9\linewidth]{./img/task2_alt_col_scheme.jpg}
\caption{Альтернативная схема раскраски (расстояние до центра)}
\end{figure}

\begin{figure}[htbp]
\centering
\includegraphics[width=.9\linewidth]{./img/task2_alt_col_scheme_plotly.png}
\caption{3D-визуализация альтернативной схемы в Plotly}
\end{figure}
\end{enumerate}
\subsubsection{Анализ результатов}
\label{sec:orgb9a95e5}

\begin{enumerate}
\item \textbf{Связь цвета и координат}: Метод RGB=XYZ интуитивно понятен и позволяет визуально оценить распределение точек в пространстве.

\item \textbf{Статичная vs интерактивная визуализация}: Matplotlib подходит для быстрых 2D-просмотров, Plotly — для детального исследования 3D-структур благодаря возможности вращения и масштабирования.

\item \textbf{Альтернативные схемы}: Раскраска по расстоянию до центра эффективно выделяет пространственную структуру облака, особенно в 3D-представлении.
\end{enumerate}
\subsection{Задание 4: Работа со скалярными полями}
\label{sec:org55e39cd}

\subsubsection{Постановка задачи}
\label{sec:orgab9246b}

Цель задания — освоить методы создания, преобразования и анализа скалярных полей, связанных с облаками точек, включая операции сглаживания, вычисления градиента, фильтрации и визуализации.
\subsubsection{Ход выполнения}
\label{sec:org2367295}

Работа проводилась с облаком точек "Рождественский медведь" (40 000 точек после субдискретизации).

\begin{figure}[htbp]
\centering
\includegraphics[width=.9\linewidth]{./img/task_4_init_vis.png}
\caption{Исходное облако точек "Рождественский медведь"}
\end{figure}
\begin{enumerate}
\item 1-3. Базовые операции со скалярными полями
\label{sec:orge87f698}

Созданы два скалярных поля:
\begin{itemize}
\item \textbf{const\textsubscript{field}}: постоянное поле со значением 10.0
\item \textbf{height}: поле высоты (координата Z)
\end{itemize}

Продемонстрированы операции:
\begin{itemize}
\item Умножение поля на число (const\textsubscript{field} × 3 = 30.0)
\item Сложение с числом (30.0 - 5 = 25.0)
\end{itemize}

Результат: постоянное поле осталось константным после операций.
\item 4. Гауссово сглаживание
\label{sec:org13cdde0}

Применено гауссово сглаживание к полю зашумлённой высоты (\textbf{noised\textsubscript{height}}). Для постоянного поля сглаживание не даёт эффекта.

\begin{figure}[htbp]
\centering
\includegraphics[width=.9\linewidth]{./img/task4_without_gauss_smoth.png}
\caption{Поле высоты без сглаживания}
\end{figure}

\begin{figure}[htbp]
\centering
\includegraphics[width=.9\linewidth]{./img/task4_with_gauss_smooth.png}
\caption{Поле высоты после гауссова сглаживания}
\end{figure}
\item 5. Вычисление градиента
\label{sec:org539a06d}

Вычислен градиент поля высоты. Для постоянного поля градиент равен нулю.

\begin{figure}[htbp]
\centering
\includegraphics[width=.9\linewidth]{./img/task4_high_field_without_grad.png}
\caption{Поле высоты}
\end{figure}

\begin{figure}[htbp]
\centering
\includegraphics[width=.9\linewidth]{./img/task4_high_field_with_grad.png}
\caption{Градиент поля высоты (почти постоянный)}
\end{figure}
\item 6. Сглаживание скользящим средним
\label{sec:org09de89d}

Применён фильтр скользящего среднего к зашумлённому полю высоты.

\begin{figure}[htbp]
\centering
\includegraphics[width=.9\linewidth]{./img/task4_ma_noised_height_without_filter.png}
\caption{Зашумлённая высота без фильтра}
\end{figure}

\begin{figure}[htbp]
\centering
\includegraphics[width=.9\linewidth]{./img/task4_ma_noised_height_with_filter.png}
\caption{Зашумлённая высота после фильтра скользящего среднего}
\end{figure}
\item 7. Преобразование скалярного поля в RGB
\label{sec:orga0639cd}

Скалярные поля преобразованы в цвета с использованием цветовой карты \textbf{viridis}.
\item 8. Статистический анализ
\label{sec:org02330a3}

Вычислены статистические параметры полей:

\begin{center}
\begin{tabular}{lrrrrr}
Поле & Среднее & Ст. отклонение & Минимум & Максимум & Медиана\\
\hline
const\textsubscript{field} & 25.000000 & 0.00000 & 25.000000 & 25.000000 & 25.0000\\
height\textsubscript{field} & -6.420886 & 111.32608 & -214.142593 & 214.158005 & -7.8196\\
\end{tabular}
\end{center}
\item 9-10. Нормализация и интерполяция
\label{sec:org40c6e5d}

Поля нормализованы в диапазон [0, 1]. Реализована интерполяция пропущенных значений (NaN) с использованием линейной интерполяции.
\item 11. Фильтрация по значению поля
\label{sec:orgf8a22b9}

Выполнена фильтрация точек по значениям скалярных полей:
\begin{itemize}
\item const\textsubscript{field}: все точки (значение 25.0 в диапазоне [0, 26])
\item height: точки с высотой от -100 до 100
\end{itemize}

После фильтрации получено подмножество точек.
\item 12. Использование скалярного поля как координаты
\label{sec:org1d13237}

Продемонстрирована возможность замены координаты Z на значение скалярного поля.

\begin{figure}[htbp]
\centering
\includegraphics[width=.9\linewidth]{./img/task4_use_scalar_as_z.png}
\caption{Использование скалярного поля как координаты Z}
\end{figure}
\item 13. Удаление скалярного поля
\label{sec:org53b7574}

Продемонстрирована операция удаления скалярного поля из облака точек.
\end{enumerate}
\subsubsection{Анализ результатов}
\label{sec:org74b9d67}

\begin{enumerate}
\item \textbf{Операции с полями}: Базовые операции (сложение, умножение) работают корректно, что подтверждает гибкость работы со скалярными полями.

\item \textbf{Сглаживание}: Гауссов фильтр и скользящее среднее эффективно уменьшают шум в данных, что особенно важно для зашумлённых измерений.

\item \textbf{Градиент}: Вычисление градиента позволяет выделять области с быстрым изменением значений поля, что полезно для анализа рельефа или других пространственных характеристик.

\item \textbf{Визуализация}: Преобразование скалярных полей в цвета с помощью цветовых карт делает данные более наглядными и интерпретируемыми.
\end{enumerate}
\subsection{Ответы на контрольные вопросы}
\label{sec:orgf8ebb9b}

\begin{enumerate}
\item \textbf{Почему для RGB значения используются числа от 0 до 1 или от 0 до 255?}
Диапазоны 0-1 и 0-255 являются стандартными представлениями интенсивности цвета. Первый удобен для вычислений с плавающей точкой, второй — для хранения и отображения.

\item \textbf{Как можно задать цвет точек на основе скалярной величины?}
Скалярное значение нормализуется в диапазон [0, 1], затем применяется цветовая карта (colormap), которая сопоставляет каждому значению цвет RGB.

\item \textbf{Чем отличается статичная визуализация (Matplotlib) от интерактивной (Plotly)?}
Matplotlib создаёт статические изображения, Plotly — интерактивные веб-визуализации с возможностью вращения, масштабирования, выделения областей.

\item \textbf{Как выбрать подходящую цветовую карту?}
Для последовательных данных (высота, температура) подходят плавные карты (viridis, plasma), для категориальных данных — дискретные карты (tab10, Set3).

\item \textbf{Что произойдёт, если не нормализовать значения при генерации цветов?}
Цвета будут некорректными: значения за пределами [0, 1] будут обрезаны, что приведёт к потере информации и искажению визуализации.
\end{enumerate}
\subsection{Выводы}
\label{sec:org666f054}

\begin{enumerate}
\item \textbf{Задание 2}: Освоены методы визуализации облаков точек с цветовой информацией. Интерактивная визуализация в Plotly оказалась значительно удобнее для исследования 3D-структур по сравнению со статичной в Matplotlib.

\item \textbf{Задание 4}: Освоены основные операции работы со скалярными полями: создание, преобразование, сглаживание, вычисление градиента, фильтрация. Эти операции являются фундаментальными для анализа пространственных данных.
\end{enumerate}
\subsection{Заключение}
\label{sec:orgc9a10a7}

В ходе выполнения заданий 2 и 4 были успешно освоены ключевые аспекты работы с облаками точек: визуализация с использованием цветовой информации и обработка скалярных полей. Реализованные методы и алгоритмы образуют базовый инструментарий для работы с трёхмерными данными, который может быть расширен и адаптирован для решения конкретных прикладных задач в компьютерном зрении, робототехнике, геодезии и других областях.
\end{document}
