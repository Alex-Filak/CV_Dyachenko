% Created 2025-12-09 Tue 19:37
% Intended LaTeX compiler: pdflatex
\documentclass[a4paper,14pt]{extarticle}
\usepackage{graphicx}
\usepackage{longtable}
\usepackage{wrapfig}
\usepackage{rotating}
\usepackage[normalem]{ulem}
\usepackage{amsmath}
\usepackage{amssymb}
\usepackage{capt-of}
\usepackage[english, main=russian, russian]{babel}
\usepackage{fontspec}
\usepackage{polyglossia}
\setdefaultlanguage{russian}
\setotherlanguage{english}
\setmainfont{CMU Serif}
\setsansfont{CMU Sans Serif}
\setmonofont{CMU Typewriter Text}
\usepackage{svg}
\usepackage{xcolor}
\usepackage{caption}
\newcommand{\incsvg}[2][]{\begin{center}\includesvg[width=\textwidth]{#2}\captionof{figure}{#1}\end{center}}
\usepackage{titlesec}
\titleformat{\section}{\normalfont\large\bfseries}{\thesection}{1em}{}
\usepackage{multicol}
\setlength{\columnsep}{1cm}
\setlength{\parindent}{0pt}
\usepackage[left=1.2cm,right=1.2cm,top=1cm,bottom=2cm]{geometry}
\linespread{1}
\usepackage{hyperref}
\usepackage{amsmath}
\usepackage{mathtools}
\usepackage{array}
\usepackage{float}
\restylefloat{table}
\renewcommand{\arraystretch}{1.7}
\date{\today}
\title{}
\hypersetup{
 pdfauthor={},
 pdftitle={},
 pdfkeywords={},
 pdfsubject={},
 pdfcreator={Emacs 29.3 (Org mode 9.8-pre)}, 
 pdflang={Russian}}
\begin{document}

\begin{center}
МФТИ\\

\vspace{6cm}



\begin{large}
\textbf{Отечет по практическим заданиям} \\
\textbf{Задание 1 и Задание 2}


\vspace{3cm}

\textbf{Филак Александр}

\end{large}
\vspace{9cm}


\vspace{2cm}

Долгопрудный \\
2025
\pagebreak
\end{center}
\section{ОТЧЁТ ПО ПРАКТИЧЕСКИМ ЗАДАНИЯМ 1 И 2}
\label{sec:org522034e}

\subsection{Введение}
\label{sec:orgeb71e2c}

В рамках двух практических заданий были изучены методы подготовки данных для машинного обучения на основе крупномасштабных облаков точек. Задание 1 было посвящено работе с датасетом Semantic3D, представляющим собой набор outdoor-сцен, а задание 2 — датасету S3DIS, содержащему indoor-сцены (помещения). Оба задания направлены на формирование единого массива данных в формате NumPy, объединяющего пространственные координаты, цветовые характеристики и метки семантических классов для последующего использования в моделях машинного обучения.

Основная цель работ — освоить полный цикл подготовки данных: от загрузки сырых файлов до формирования нормализованных массивов, готовых для подачи в нейронные сети. Особое внимание уделено особенностям каждого датасета, проблемам предобработки и способам их решения.
\subsection{Теоретическая часть}
\label{sec:org2385049}

\subsubsection{Датасет Semantic3D}
\label{sec:org2687124}

Semantic3D — это крупномасштабный датасет для семантической сегментации облаков точек, содержащий более 4 миллиардов точек, размеченных по 8 классам. Данные получены с помощью наземного лидара в различных outdoor-сценах: городские площади, улицы, исторические здания. Датасет характеризуется высокой плотностью точек, наличием шума и неравномерным распределением классов.
\subsubsection{Датасет S3DIS}
\label{sec:orga257294}

S3DIS (Stanford 3D Indoor Spaces Dataset) — датасет для семантической сегментации внутренних помещений. Он содержит 6 крупных областей (зданий), разделённых на 271 помещение, с размеченными по 13 классам объектами (стены, пол, потолок, мебель и т.д.). Данные собраны с помощью сканера Matterport и имеют более структурированный характер по сравнению с Semantic3D.
\subsubsection{Предобработка облаков точек}
\label{sec:org592c265}

Ключевые этапы предобработки включают:
\begin{itemize}
\item Нормализацию координат для устранения смещений и масштабных различий
\item Преобразование цветов из диапазона [0, 255] в [0, 1]
\item Формирование единой таблицы данных с признаками и метками
\item Визуализацию распределения классов для анализа сбалансированности
\end{itemize}
\subsection{Задание 1: Подготовка данных Semantic3D}
\label{sec:org1de3345}

\subsubsection{Постановка задачи}
\label{sec:org3efa3ff}

Цель задания — загрузить данные из файла датасета Semantic3D (например, `bildstein\textsubscript{station1}\textsubscript{xyz}\textsubscript{intensity}\textsubscript{rgb.label}`), сформировать единый массив NumPy, содержащий координаты (X, Y, Z), интенсивность, цвета (R, G, B) и метку класса, выполнить нормализацию данных и сохранить результат в форматах .npy, .txt и .h5.
\subsubsection{Ход выполнения}
\label{sec:org63dfd2f}

Полный код реализации и результаты выполнения представлены в приложенном Jupyter Notebook (PDF-файл). Ниже описаны ключевые этапы:

\begin{enumerate}
\item \textbf{Загрузка данных}: Использована функция \texttt{np.loadtxt()} для чтения текстового файла с разделителем-пробелом.

\item \textbf{Извлечение признаков}: Из массива выделены столбцы с координатами, интенсивностью, цветами и меткой класса.

\item \textbf{Нормализация}:
\begin{itemize}
\item Координаты центрированы и масштабированы
\item Цвета преобразованы из диапазона [0, 255] в [0, 1]
\item Интенсивность нормализована
\end{itemize}

\item \textbf{Формирование таблицы}: Все признаки объединены в единый массив размером N×8, где N — количество точек.

\item \textbf{Сохранение}: Данные сохранены в файлы `semantic3d\textsubscript{dataset.npy}`, `semantic3d\textsubscript{dataset.txt}` и `semantic3d\textsubscript{dataset.h5}`.

\item \textbf{Визуализация}: Построена гистограмма распределения меток классов.
\end{enumerate}
\subsubsection{Ответы на контрольные вопросы}
\label{sec:org54f63a5}

\begin{enumerate}
\item \textbf{Semantic3D} — это датасет для семантической сегментации outdoor-сцен, содержащий облака точек, полученные с помощью наземного лидара. Он включает 8 классов: здания, дороги, растительность и др.

\item \textbf{Отличие от 2D-датасетов}: Semantic3D представляет трёхмерные данные с пространственной информацией, в отличие от ImageNet, который содержит 2D-изображения. От S3DIS отличается outdoor-сценами vs indoor.

\item \textbf{Состав датасета}: Разделён на train (15 сцен) и test (15 сцен). Train используется для обучения, test — для оценки.

\item \textbf{Технология получения}: Данные получены с помощью лидара, что приводит к высокой плотности точек, но также к шуму и неравномерному покрытию.

\item \textbf{8 классов}: building, road, vegetation, trunk, pole, car, fencing, other. Например, building — здания, road — дороги.

\item \textbf{Проблемы разметки}: Шум, артефакты, неоднозначность границ между классами.

\item \textbf{Технические проблемы}: Большой объём данных требует значительных вычислительных ресурсов для обработки и хранения.

\item \textbf{Неравномерная плотность}: В лидарных данных плотность точек уменьшается с расстоянием, что может привести к плохой сегментации удалённых объектов.

\item \textbf{Несбалансированность классов}: Некоторые классы (например, vegetation) представлены значительно больше, чем другие (pole), что может смещать модель в сторону частых классов.

\item \textbf{Этапы предобработки}: Нормализация координат, даунсамплинг, фильтрация шума, балансировка классов.
\end{enumerate}
\subsection{Задание 2: Подготовка данных S3DIS}
\label{sec:orgc5ef2cf}

\subsubsection{Постановка задачи}
\label{sec:org7aa4cd2}

Цель задания — загрузить данные из датасета S3DIS, сформировать массив для каждого помещения всех областей, содержащий координаты, цвета и метки классов, выполнить нормализацию и сохранить результат.
\subsubsection{Ход выполнения}
\label{sec:org6f57681}

Реализация представлена в Jupyter Notebook (PDF-файл). Основные этапы:

\begin{enumerate}
\item \textbf{Загрузка данных}: Рекурсивный обход каталогов датасета и чтение .txt файлов.

\item \textbf{Обработка каждого помещения}: Для каждого файла выделены координаты, цвета и метки.

\item \textbf{Нормализация}: Координаты нормализованы относительно границ помещения, цвета приведены к диапазону [0, 1].

\item \textbf{Объединение данных}: Сформирован единый массив для всех помещений выбранной области.

\item \textbf{Сохранение}: Данные сохранены в форматах .npy и .h5 для последующего использования.

\item \textbf{Визуализация}: Построены гистограммы распределения классов по помещениям.
\end{enumerate}
\subsubsection{Ответы на контрольные вопросы}
\label{sec:org9e14cdb}

\begin{enumerate}
\item \textbf{S3DIS} — датасет для семантической сегментации внутренних помещений, содержащий 271 комнату с размеченными объектами.

\item \textbf{Типы данных}: Облака точек с координатами, цветами и метками классов.

\item \textbf{6 областей}: Area 1–6, каждая содержит несколько зданий.

\item \textbf{Формат файла}: Текстовый файл с колонками: X Y Z R G B label.

\item \textbf{Задачи ML}: Семантическая сегментация, классификация объектов, кластеризация.

\item \textbf{13 классов}: ceiling, floor, wall, beam, column, window, door, table, chair, sofa, bookcase, board, clutter.

\item \textbf{Признаки для обучения}: Координаты, цвета, нормали, локальные дескрипторы.

\item \textbf{Нормализация координат}: Необходима для устранения смещений и приведения данных к единому масштабу, что улучшает сходимость моделей.

\item \textbf{Сохранение в NumPy}: Форматы .npy/.npz обеспечивают быструю загрузку и эффективное хранение данных.

\item \textbf{Разделение признаков и меток}: Последний столбец массива — метка класса, остальные — признаки.

\item \textbf{Отличие функций сохранения}: \texttt{np.save()} сохраняет один массив, \texttt{np.savez()} — несколько, \texttt{np.savez\_compressed()} — сжатые данные.

\item \textbf{Потенциальные проблемы}: Большой объём данных, несбалансированность классов, пропуски в данных.
\end{enumerate}
\subsection{Выводы}
\label{sec:org5b2c7fe}

\begin{enumerate}
\item \textbf{Задание 1}: Освоены методы работы с крупномасштабными outdoor-облаками точек. Особое внимание уделено обработке лидарных данных, характеризующихся неравномерной плотностью и шумом.

\item \textbf{Задание 2}: Освоена подготовка данных для indoor-сцен, где важную роль играет структурная информация о помещениях и чёткое разделение на классы объектов интерьера.
\end{enumerate}
\subsection{Заключение}
\label{sec:orgd09408d}

В ходе выполнения заданий были успешно подготовлены данные из датасетов Semantic3D и S3DIS для использования в задачах машинного обучения. Реализован полный конвейер обработки: от загрузки сырых данных до формирования нормализованных массивов NumPy. Особое внимание было уделено ответам на контрольные вопросы, которые позволили глубже понять особенности каждого датасета и связанные с ними проблемы. Полученные массивы данных готовы для использования в обучении моделей семантической сегментации облаков точек.
\end{document}
