% Created 2025-12-09 Tue 17:54
% Intended LaTeX compiler: pdflatex
\documentclass[a4paper,14pt]{extarticle}
\usepackage{graphicx}
\usepackage{longtable}
\usepackage{wrapfig}
\usepackage{rotating}
\usepackage[normalem]{ulem}
\usepackage{amsmath}
\usepackage{amssymb}
\usepackage{capt-of}
\usepackage[english, main=russian, russian]{babel}
\usepackage{fontspec}
\usepackage{polyglossia}
\setdefaultlanguage{russian}
\setotherlanguage{english}
\setmainfont{CMU Serif}
\setsansfont{CMU Sans Serif}
\setmonofont{CMU Typewriter Text}
\usepackage{svg}
\usepackage{xcolor}
\usepackage{caption}
\newcommand{\incsvg}[2][]{\begin{center}\includesvg[width=\textwidth]{#2}\captionof{figure}{#1}\end{center}}
\usepackage{titlesec}
\titleformat{\section}{\normalfont\large\bfseries}{\thesection}{1em}{}
\usepackage{multicol}
\setlength{\columnsep}{1cm}
\setlength{\parindent}{0pt}
\usepackage[left=1.2cm,right=1.2cm,top=1cm,bottom=2cm]{geometry}
\linespread{1}
\usepackage{hyperref}
\usepackage{amsmath}
\usepackage{mathtools}
\usepackage{array}
\usepackage{float}
\restylefloat{table}
\renewcommand{\arraystretch}{1.7}
\date{\today}
\title{}
\hypersetup{
 pdfauthor={},
 pdftitle={},
 pdfkeywords={},
 pdfsubject={},
 pdfcreator={Emacs 29.3 (Org mode 9.8-pre)}, 
 pdflang={Russian}}
\begin{document}

\begin{center}
МФТИ\\

\vspace{6cm}



\begin{large}
\textbf{Отечет по практическим заданиям} \\
\textbf{1b и 2b}


\vspace{3cm}

\textbf{Филак Александр}

\end{large}
\vspace{9cm}


\vspace{2cm}

Долгопрудный \\
2025
\pagebreak
\end{center}
\section{Введение}
\label{sec:orgbb4adf1}

В рамках двух практических заданий были изучены и реализованы методы обработки облаков точек с использованием Python и библиотеки NumPy. Задание 1b было посвящено методам субдискретизации (downsampling) облаков точек, а задание 2b — методам пространственной фильтрации и выделения областей по геометрическим признакам. Оба задания направлены на освоение базовых операций предобработки точечных данных, которые являются критически важными для последующего анализа в компьютерном зрении, робототехнике и 3D-моделировании.
\section{Задание 1b: Субдискретизация облаков точек}
\label{sec:orgef36379}

Целью данного задания было освоение различных методов уменьшения плотности облаков точек при сохранении их ключевых геометрических характеристик.
\subsection{Методы реализации}
\label{sec:org43627cb}

Были реализованы три метода субдискретизации:

\begin{enumerate}
\item \textbf{Случайная выборка (Random subsampling)} — равномерный случайный выбор N точек из исходного облака. Наиболее быстрый, но может терять важные детали.

\item \textbf{Воксельная сетка (Voxel grid subsampling)} — пространство делится на равные ячейки (вокселы), в каждой из которых остаётся только одна точка. Эффективно уменьшает плотность в равномерно заполненных областях.

\item \textbf{Выбор наиболее удалённых точек (Farthest Point Sampling)} — итеративный выбор точек, максимально удалённых от уже выбранных. Наиболее эффективно сохраняет геометрию, но требует больше вычислений.
\end{enumerate}
\subsection{Визуализация результатов}
\label{sec:org83f2741}

\begin{figure}[htbp]
\centering
\includegraphics[width=.9\linewidth]{./img/orig_pc.png}
\caption{Исходное облако точек}
\end{figure}

\begin{figure}[htbp]
\centering
\includegraphics[width=.9\linewidth]{./img/random_subsample.png}
\caption{Случайная субдискретизация (50\%)}
\end{figure}

\begin{figure}[htbp]
\centering
\includegraphics[width=.9\linewidth]{./img/voxel_subsample.png}
\caption{Воксельная субдискретизация (размер вокселя 0.001)}
\end{figure}

\begin{figure}[htbp]
\centering
\includegraphics[width=.9\linewidth]{./img/fps_subsample.png}
\caption{Выбор наиболее удалённых точек (FPS)}
\end{figure}
\subsection{Анализ результатов}
\label{sec:org40d39e7}

\begin{enumerate}
\item \textbf{Влияние размера выборки на визуальное качество}: При уменьшении выборки до 50\% случайным методом заметна потеря мелких деталей и появление "дыр" в равномерных областях.

\item \textbf{Влияние размера вокселя}: При размере вокселя 0.001 сохраняется больше деталей по сравнению с бóльшими размерами. Мелкий воксель лучше сохраняет геометрию, но даёт меньшую степень сжатия.

\item \textbf{Эффективность по времени}: Случайная выборка — самый быстрый метод (\textasciitilde{}0.001 с), воксельная сетка
(\textasciitilde{}1.0 с) и FPS (\textasciitilde{}1.0 с) — существенно более медленные.

\item \textbf{Сохранение геометрии}: FPS лучше всего сохраняет геометрические особенности объекта, воксельная сетка хорошо сохраняет равномерное покрытие, случайная выборка может терять важные детали.
\end{enumerate}
\section{Задание 2b: Выделение областей в облаках точек}
\label{sec:orgfc73515}

Целью данного задания было освоение методов пространственной фильтрации точек по различным геометрическим критериям.
\subsection{Методы реализации}
\label{sec:org9fc0bc0}

Были реализованы три метода фильтрации:

\begin{enumerate}
\item \textbf{Ограничивающий параллелепипед (Bounding Box)} — выделение точек, попадающих в заданный диапазон координат.

\item \textbf{Фильтр по высоте} — выделение точек выше определённого уровня (Z > порогового значения).

\item \textbf{Фильтр по радиусу} — выделение точек в пределах заданного радиуса от центральной точки.
\end{enumerate}
\subsection{Визуализация результатов}
\label{sec:org201380b}

\begin{figure}[htbp]
\centering
\includegraphics[width=.9\linewidth]{./img/bbox_subsample.png}
\caption{Выделение области ограничивающим параллелепипедом}
\end{figure}

\begin{figure}[htbp]
\centering
\includegraphics[width=.9\linewidth]{./img/height_restriction.png}
\caption{Фильтрация по высоте (верхние точки)}
\end{figure}

\begin{figure}[htbp]
\centering
\includegraphics[width=.9\linewidth]{./img/in_radius.png}
\caption{Выделение точек в заданном радиусе}
\end{figure}
\section{Выводы}
\label{sec:orgb9140fd}

\begin{enumerate}
\item Для быстрого уменьшения объёма данных без сохранения точной геометрии достаточно случайной выборки.

\item Для сохранения геометрических особенностей при уменьшении плотности оптимален метод FPS, несмотря на его вычислительную сложность.

\item Воксельная сетка представляет собой хороший компромисс между скоростью и сохранением структуры.

\item Пространственная фильтрация позволяет эффективно выделять значимые области для последующего анализа.

\item Сочетание методов субдискретизации и фильтрации позволяет создавать гибкие конвейеры предобработки 3D-данных для различных прикладных задач.
\end{enumerate}
\section{Заключение}
\label{sec:org9558079}

В ходе выполнения заданий были успешно освоены базовые методы обработки облаков точек с использованием Python и NumPy. Полученные навыки субдискретизации и пространственной фильтрации являются фундаментальными для работы с 3D-данными в таких областях, как компьютерное зрение, робототехника, автономные системы и 3D-моделирование. Реализованные алгоритмы могут быть использованы как самостоятельные инструменты, так и в качестве этапов предобработки в более сложных конвейерах обработки точечных данных.
\end{document}
